
# This file will be loaded in latex. Useful to pass data to the main document
f_out_write = open("robExt-D7A1E2B137A7DE888432061B49A70C15-out.tex", "w")

import os
import sys

def write_to_out(text):
    """Write to the -out.tex file that is loaded by default"""
    f_out_write.write(text)

def parse_args():
    args = {}
    if len(sys.argv) % 2 == 0:
        print("Error: the number of arguments must be even, as tuples of name and value")
        exit(1)
    for i in range(0,len(sys.argv)-1,2):
        args[sys.argv[i+1]] = sys.argv[i+2]
    return args

def get_cache_folder():
    '''
    Path of the cache folder. Warning: this works only when the python script
    is located in this cache folder (that should be true when it's called from LaTeX)
    '''
    return os.path.abspath(os.path.dirname(sys.argv[0]))

def get_file_base():
    '''
    Outputs the base of the files (i.e. something like robExt-somehash, without any extension)
    '''
    return os.path.splitext(os.path.basename(sys.argv[0]))[0] # __file__ does not work as it refers to the library

def get_current_script():
    '''
    Outputs the path of the current script
    '''
    return os.path.abspath(sys.argv[0]) # __file__ does not work as it refers to the library


def get_filename_from_extension(extension):
    '''
    If you want to create a file with extension 'extension' (with the appropriate base name), this command
    is for you. For instance get_filename_from_extension(".mp4") would return something like
    robExt-somehash.mp4
    the extension can also be like get_filename_from_extension("-out.tex") etc.
    '''
    return os.path.join(get_cache_folder(), get_file_base() + extension)

def get_verbatim_output():
    '''Returns the path to -out.txt that is read by verbatim output'''
    return get_filename_from_extension("-out.txt")

def get_pdf_output():
    '''Returns the path to -out.txt that is read by verbatim output'''
    return get_filename_from_extension(".pdf")


def finished_with_no_error():
    '''
    Call this at the end of your script. This creates the path of the final pdf file that should be
    created (otherwise robust-externalize will think that the compilation failed)
    '''
    if not os.path.exists(get_filename_from_extension(".pdf")):
        # we create a nearly empty pdf (not empty or arxiv will remove it)
        with open(get_filename_from_extension(".pdf"), 'w') as f:
            f.write("ok")

### Starting main content
import matplotlib.pyplot as plt
import matplotlib
import numpy as np
import json
matplotlib.use("pgf")
#### See this link for details on how to preview the image in jupyter
#### https://matplotlib.org/stable/users/explain/text/pgf.html
matplotlib.rcParams.update({
  "font.family": "serif",
  "font.serif": [], # Use LaTeX default serif font.
  "text.usetex": True, # use inline math for ticks
  #### You can change the font size of individual items with:
  #### "font.size": 11,
  #### "axes.titlesize": 11,
  #### "legend.fontsize": 11,
  #### "axes.labelsize": 11,
})

with open('../data1.json') as f:
    stats = json.load(f)
fig, ax = plt.subplots()
fig.set_size_inches(4.25197, 0.7*4.25197, forward=True)
cases = list({case_name for program_label, program_stats in stats.items() for case_name in program_stats})
width = 0.25
multiplier = 0
x = np.arange(len(cases))
for program_label, program_stats in stats.items():
    offset = width * multiplier
    rects = ax.bar(x + offset, [program_stats[case]["total_execution"]["t_total"] for case in cases], width, label=program_label)
    ax.bar_label(rects, padding=3)
    multiplier += 1

ax.set_xticks(x + width/2)
ax.set_xticklabels([i for i in range(len(cases))])
ax.set_ylabel('Execution time [ms]')
ax.set_title("Total execution time")
ax.legend(title='Program')

print(get_filename_from_extension(".pgf"))
plt.savefig(get_filename_from_extension(".pgf"), bbox_inches="tight")




### Ending main content
finished_with_no_error()


f_out_write.close()




