% type: problems
% title: Teoria informacji kolokwium 4 grudnia 2024
% created: 2024-12-08
\input{vars.tex}

\newcommand{\nop}[1]{}

\def\gettexliveversion#1(#2 #3 #4#5#6#7#8)#9\relax{#4#5#6#7}
\edef\texliveversion{\expandafter\gettexliveversion\pdftexbanner\relax}

\def\bthandled{0}
\def\btunhandled{0}
\def\btproblems{problems}%
\def\bttechnicalslides{technicalslides}%

\ifx\buildtype\btproblems
    \documentclass{article}
    \def\bthandled{1}
    \usepackage{authblk}
    \usepackage[a4paper, total={6in, 8in}]{geometry}
    \author[1]{\buildauthor}
    \affil[1]{\buildinstitute}
    \date{\buildcreated\hspace*{0.5cm}(updated: \builddate)}
    \title{\buildtitle}

    \def\buildpdftitle{\buildtitle}
    \def\buildpdfsubject{Simple short report for classes}
    \def\buildpdfauthor{\buildauthor}
    \def\buildpdfkeywords{}
    \def\buildpdfcreator{\buildauthor - \buildinstitute}
    \def\buildpdfproducer{TeXLive=\texliveversion; CMake=\buildcmakeversion; Template=\buildtype(article)}

    \usepackage{lipsum}
\usepackage{tikz}
\usepackage{shellesc}
\usepackage{pdftexcmds}
\usepackage{robust-externalize}
\usepackage{relsize}
\usepackage{algorithm}
\PassOptionsToPackage{noend}{algpseudocode}
\usepackage{algpseudocode}
\usepackage{float}
\usepackage{mathtools}
\usepackage{caption}
\usepackage[dvipsnames, x11names, svgnames]{xcolor}
\usepackage{colortbl}
\usepackage{color,soul}
\usepackage{minted}
\usepackage{filecontents}
\usepackage[backend=biber]{biblatex}


\addbibresource{main.bib}

\pgfmathsetseed{\number\pdfrandomseed}
\usetikzlibrary{tikzmark, positioning, decorations.pathreplacing, calc, arrows.meta, shapes.geometric, backgrounds, arrows}

\robExtConfigure{
  compile in parallel with gnu parallel
}

\def\mathdefault#1{#1} % Needed in matplotlib 3.8: https://github.com/matplotlib/matplotlib/issues/27907
\setbeamertemplate{frametitle}[default][center]
% Removes icon in bibliography
\setbeamertemplate{bibliography item}{}

\renewcommand{\algorithmiccomment}[1]{\hfill$\triangleright$\textit{#1}}
\newcommand{\CommentH}[1]{\Comment{\textbf{\textcolor{BlueViolet}{#1}}}}

\tikzset{fontscale/.style = {font=\relsize{#1}}}

\DeclarePairedDelimiter\ceil{\lceil}{\rceil}
\DeclarePairedDelimiter\floor{\lfloor}{\rfloor}

%% FIXES FOR ALGORITHMS

\usepackage{etoolbox}

\newcommand{\algruledefaultfactor}{.75}
\newcommand{\algstrut}[1][\algruledefaultfactor]{\vrule width 0pt
depth .25\baselineskip height #1\baselineskip\relax}
\newcommand*{\algrule}[1][\algorithmicindent]{\hspace*{.5em}\vrule\algstrut
\hspace*{\dimexpr#1-.5em}}

\makeatletter
\newcount\ALG@printindent@tempcnta
\def\ALG@printindent{%
    \ifnum \theALG@nested>0% is there anything to print
    \ifx\ALG@text\ALG@x@notext% is this an end group without any text?
    % do nothing
    \else
    \unskip
    % draw a rule for each indent level
    \ALG@printindent@tempcnta=1
    \loop
    \algrule[\csname ALG@ind@\the\ALG@printindent@tempcnta\endcsname]%
    \advance \ALG@printindent@tempcnta 1
    \ifnum \ALG@printindent@tempcnta<\numexpr\theALG@nested+1\relax% can't do <=, so add one to RHS and use < instead
    \repeat
    \fi
    \fi
}%

\patchcmd{\ALG@doentity}{\noindent\hskip\ALG@tlm}{\ALG@printindent}{}{\errmessage{failed to patch}}

\AtBeginEnvironment{algorithmic}{\lineskip0pt}

\newcommand*\Let[2]{\State #1 $\gets$ #2}
\newcommand*\Stateh{\State \algstrut[1]}


%% END FIXES
%% COLOURS FOR ALGORITHMS


\makeatletter
\newcommand{\algcolor}[2]{%
  \hskip-\ALG@thistlm\colorbox{#1}{\parbox{\dimexpr\linewidth-2\fboxsep}{\hskip\ALG@thistlm\relax #2}}%
}
\newcommand{\algemph}[1]{\algcolor{GreenYellow}{#1}}
\makeatother

% END COLOURS


\newcommand{\randeq}[1]{% 
\pgfmathparse{(int(random(-40, 40))+100)/100 * #1}%
\pgfmathresult%
}%

\tikzset{
    bigbox/.style={draw, rounded corners, minimum width=1.5cm, minimum height=1cm},
    smallbox/.style={draw, rounded corners, minimum width=1.25cm, minimum height=0.75cm},
    tinybox/.style={draw, rounded corners, minimum width=1.25cm, minimum height=0.6cm},
    bigcircle/.style={draw, circle, minimum size=1cm},
    bigellipse/.style={draw, ellipse, minimum width=1.5cm, minimum height=1.25cm},
    place/.style={inner sep=0pt, outer sep=0pt},
    colprimaryl/.style={draw=NavyBlue, fill=LightSkyBlue, text=black},
    colprimary/.style={draw=NavyBlue, fill=NavyBlue, text=white},
    operation/.style={draw=FireBrick, fill=LightSalmon, text=black, rounded corners},
    fork/.style={decorate, decoration={show path construction, lineto code={
        \draw[->](\tikzinputsegmentfirst)-|($(\tikzinputsegmentfirst)!.5!(\tikzinputsegmentlast)$)|-(\tikzinputsegmentlast);}
    }},
    center coordinate/.style={
        execute at end picture={
        \path ([rotate around={180:#1}]perpendicular cs: horizontal line through={#1},
                                    vertical line through={(current bounding box.east)})
                ([rotate around={180:#1}]perpendicular cs: horizontal line through={#1},
                                    vertical line through={(current bounding box.west)});}}
}

\let\oldtikzpicture\tikzpicture
\let\oldendtikzpicture\endtikzpicture

\renewenvironment{tikzpicture}{%
    \ifbuilddraft\comment%
    \else\expandafter\oldtikzpicture%   
    \fi
}{%
    \ifbuilddraft\endcomment%
    \else\oldendtikzpicture%
    \fi
}


    \begin{document}
    \maketitle
\fi

\ifx\buildtype\bttechnicalslides
    \documentclass{beamer}
    \def\bthandled{1}
    \author{\buildauthor}
    \institute{\buildinstitute}
    \date{\buildcreated\hspace*{0.5cm}(updated: \builddate)}
    \title{\buildtitle}

    \def\buildpdftitle{\buildtitle}
    \def\buildpdfsubject{Technical slide deck}
    \def\buildpdfauthor{\buildauthor}
    \def\buildpdfkeywords{}
    \def\buildpdfcreator{\buildauthor - \buildinstitute}
    \def\buildpdfproducer{TeXLive=\texliveversion; CMake=\buildcmakeversion; Template=\buildtype(beamer)}

    \makeatletter
        \ifbuildhasbibliography
            \usepackage[backend=biber, style=authortitle]{biblatex}
            \addbibresource{main.bib}
            \AtEveryCitekey{\iffootnote{\scriptsize}{\footnotesize}}
            \setbeamertemplate{navigation symbols}{}
        \fi
        \setbeamertemplate{frametitle}[default][center]
    \makeatother

    \usepackage{lipsum}
\usepackage{tikz}
\usepackage{shellesc}
\usepackage{pdftexcmds}
\usepackage{robust-externalize}
\usepackage{relsize}
\usepackage{algorithm}
\PassOptionsToPackage{noend}{algpseudocode}
\usepackage{algpseudocode}
\usepackage{float}
\usepackage{mathtools}
\usepackage{caption}
\usepackage[dvipsnames, x11names, svgnames]{xcolor}
\usepackage{colortbl}
\usepackage{color,soul}
\usepackage{minted}
\usepackage{filecontents}
\usepackage[backend=biber]{biblatex}


\addbibresource{main.bib}

\pgfmathsetseed{\number\pdfrandomseed}
\usetikzlibrary{tikzmark, positioning, decorations.pathreplacing, calc, arrows.meta, shapes.geometric, backgrounds, arrows}

\robExtConfigure{
  compile in parallel with gnu parallel
}

\def\mathdefault#1{#1} % Needed in matplotlib 3.8: https://github.com/matplotlib/matplotlib/issues/27907
\setbeamertemplate{frametitle}[default][center]
% Removes icon in bibliography
\setbeamertemplate{bibliography item}{}

\renewcommand{\algorithmiccomment}[1]{\hfill$\triangleright$\textit{#1}}
\newcommand{\CommentH}[1]{\Comment{\textbf{\textcolor{BlueViolet}{#1}}}}

\tikzset{fontscale/.style = {font=\relsize{#1}}}

\DeclarePairedDelimiter\ceil{\lceil}{\rceil}
\DeclarePairedDelimiter\floor{\lfloor}{\rfloor}

%% FIXES FOR ALGORITHMS

\usepackage{etoolbox}

\newcommand{\algruledefaultfactor}{.75}
\newcommand{\algstrut}[1][\algruledefaultfactor]{\vrule width 0pt
depth .25\baselineskip height #1\baselineskip\relax}
\newcommand*{\algrule}[1][\algorithmicindent]{\hspace*{.5em}\vrule\algstrut
\hspace*{\dimexpr#1-.5em}}

\makeatletter
\newcount\ALG@printindent@tempcnta
\def\ALG@printindent{%
    \ifnum \theALG@nested>0% is there anything to print
    \ifx\ALG@text\ALG@x@notext% is this an end group without any text?
    % do nothing
    \else
    \unskip
    % draw a rule for each indent level
    \ALG@printindent@tempcnta=1
    \loop
    \algrule[\csname ALG@ind@\the\ALG@printindent@tempcnta\endcsname]%
    \advance \ALG@printindent@tempcnta 1
    \ifnum \ALG@printindent@tempcnta<\numexpr\theALG@nested+1\relax% can't do <=, so add one to RHS and use < instead
    \repeat
    \fi
    \fi
}%

\patchcmd{\ALG@doentity}{\noindent\hskip\ALG@tlm}{\ALG@printindent}{}{\errmessage{failed to patch}}

\AtBeginEnvironment{algorithmic}{\lineskip0pt}

\newcommand*\Let[2]{\State #1 $\gets$ #2}
\newcommand*\Stateh{\State \algstrut[1]}


%% END FIXES
%% COLOURS FOR ALGORITHMS


\makeatletter
\newcommand{\algcolor}[2]{%
  \hskip-\ALG@thistlm\colorbox{#1}{\parbox{\dimexpr\linewidth-2\fboxsep}{\hskip\ALG@thistlm\relax #2}}%
}
\newcommand{\algemph}[1]{\algcolor{GreenYellow}{#1}}
\makeatother

% END COLOURS


\newcommand{\randeq}[1]{% 
\pgfmathparse{(int(random(-40, 40))+100)/100 * #1}%
\pgfmathresult%
}%

\tikzset{
    bigbox/.style={draw, rounded corners, minimum width=1.5cm, minimum height=1cm},
    smallbox/.style={draw, rounded corners, minimum width=1.25cm, minimum height=0.75cm},
    tinybox/.style={draw, rounded corners, minimum width=1.25cm, minimum height=0.6cm},
    bigcircle/.style={draw, circle, minimum size=1cm},
    bigellipse/.style={draw, ellipse, minimum width=1.5cm, minimum height=1.25cm},
    place/.style={inner sep=0pt, outer sep=0pt},
    colprimaryl/.style={draw=NavyBlue, fill=LightSkyBlue, text=black},
    colprimary/.style={draw=NavyBlue, fill=NavyBlue, text=white},
    operation/.style={draw=FireBrick, fill=LightSalmon, text=black, rounded corners},
    fork/.style={decorate, decoration={show path construction, lineto code={
        \draw[->](\tikzinputsegmentfirst)-|($(\tikzinputsegmentfirst)!.5!(\tikzinputsegmentlast)$)|-(\tikzinputsegmentlast);}
    }},
    center coordinate/.style={
        execute at end picture={
        \path ([rotate around={180:#1}]perpendicular cs: horizontal line through={#1},
                                    vertical line through={(current bounding box.east)})
                ([rotate around={180:#1}]perpendicular cs: horizontal line through={#1},
                                    vertical line through={(current bounding box.west)});}}
}

\let\oldtikzpicture\tikzpicture
\let\oldendtikzpicture\endtikzpicture

\renewenvironment{tikzpicture}{%
    \ifbuilddraft\comment%
    \else\expandafter\oldtikzpicture%   
    \fi
}{%
    \ifbuilddraft\endcomment%
    \else\oldendtikzpicture%
    \fi
}


    \begin{document}
    \frame{\titlepage}
\fi


\ifx\bthandled\btunhandled
    % Throw error normally
    \PackageError{preamble}{Unrecognized type of document: '\buildtype'}{}
    % change interaction mode (if not yet in batchmode)
    \batchmode
    % ask for terminal input (which would be stored in the macro `\foo`)
    \read-1 to \foo
\fi



\section{Problem 1}

Intuitivelly choosing $F$ gives us the values for each $X$ while $X$ only says about some of the values.
We can construct such set $L$ that $f_{i}(x) = x \iff x \leq k$ for $k = \inf\{ k : (n-k)! \geq n \}$
For such a contruct $X=x \land x \leq k \implies Y=y=x$
If $k=n$ then we have just $X=x \implies Y=y$ 
That is discussed later in examples section.


Now let's take a look at the fact that:
\begin{equation}
\begin{cases}
    \pri{Y;X} = \pre{Y} - \pre{Y}[X] \\
    \pri{Y;F} = \pre{Y} - \pre{Y}[F]
\end{cases}
\end{equation}

\begin{gather*}
\pri{Y;X} \leq \pri{Y;F} \\
- \pre{Y}[X] \leq - \pre{Y}[F] 
\end{gather*}

\begin{gather*}
\sum_{x}{\prob{X} * \sum_{y}{ \prob{Y=y}[X=x] * \log{\prob{Y=y}[X=x]} }} \leq \\
\leq \sum_{f}{\prob{F} * \sum_{y}{ \prob{Y=y}[F=f] * \log{\prob{Y=y}[F=f]} }} 
\end{gather*}

We can get rid of any of the $\prob{X}$ and $\prob{F}$:

\begin{gather*}
\sum_{x}{\sum_{y}{ \prob{Y=y}[X=x] * \log{\prob{Y=y}[X=x]} }} \leq \\
\leq \sum_{f}{\sum_{y}{ \prob{Y=y}[F=f] * \log{\prob{Y=y}[F=f]} }} 
\end{gather*}

We expand the definition using $\prob{b} = \sum_a{\prob{a} * \prob{b}[a]}$:
% \begin{gather*}
% \sum_{x}{\sum_{y}{\{ (\sum_{f}{\prob{F=f}[X=x, Y=y]*[f(x)=y]}) * \log{\sum_{f}{\prob{F=f}[X=x, Y=y]*[f(x)=y]}} \}}} \leq \\
% \leq \sum_{f}{\sum_{y}{\{ (\sum_{x}{\prob{X=x}[F=f, Y=y]*[f(x)=y]}) * \log{\sum_{x}{\prob{X=x}[F=f, Y=y]*[f(x)=y]}} \}}} 
% \end{gather*}

% \begin{gather*}
% \sum_{x}{\sum_{y}{\{ (\sum_{f}{\prob{F=f}[f(x)=y]}) * \log{\sum_{f}{\prob{F=f}[f(x)=y]}} \}}} \leq \\
% \leq \sum_{f}{\sum_{y}{\{ (\sum_{x}{\prob{X=x}[f(1)=x_1, ..., f(n)=x_n]}) * \log{\sum_{x}{\prob{X=x}[f(1)=x_1, ..., f(n)=x_n]}} \}}} 
% \end{gather*}

% \begin{gather*}
% \sum_{x}{\sum_{y}{\{ (\sum_{f}{\prob{F=f}[f(x)=y]}) * \log{\sum_{f}{\prob{F=f}[f(x)=y]}} \}}} \leq \\
% \leq \sum_{f}{\sum_{y}{\{ (\sum_{x}{\prob{X=x}[f(x)=y]}) * \log{\sum_{x}{\prob{X=x}[f(1)=x_1, ..., f(n)=x_n]}} \}}} 
% \end{gather*}

% We can use the Golden lemma now 

\begin{gather*}
\sum_{x}{\sum_{y}{\{ (\sum_{f}{\prob{F}*\prob{Y=y}[X=x, F=f]}) * \log{\sum_{f}{\prob{F}*\prob{Y=y}[X=x, F=f]}} \}}} \leq \\
\leq \sum_{f}{\sum_{y}{\{ (\sum_{x}{\prob{X}*\prob{Y=y}[F=f, X=x]}) * \log{\sum_{x}{\prob{X}*\prob{Y=y}[F=f, X=x]}} \}}} 
\end{gather*}

We can get rid of $\prob{F}$ and $\prob{X}$:

\begin{gather*}
\sum_{x}{\sum_{y}{\{ (\sum_{f}{\prob{Y=y}[X=x, F=f]}) * \log{\sum_{f}{\prob{F}*\prob{Y=y}[X=x, F=f]}} \}}} \leq \\
\leq \sum_{f}{\sum_{y}{\{ (\sum_{x}{\prob{Y=y}[F=f, X=x]}) * \log{\sum_{x}{\prob{X}*\prob{Y=y}[F=f, X=x]}} \}}} 
\end{gather*}

If we define $a_{x,y,z} \defeq \prob{Y=y}[X=x, F=f]$ and $k=1/n$ rename $z$ as follows: $z \defeq f$, we get:

\begin{gather*}
\sum_{x}{\sum_{y}{\{ (\sum_{z}{a_{x,y,z}}) * \log{\sum_{z}{k*a_{x,y,z}}} \}}} \leq \\
\leq \sum_{z}{\sum_{y}{\{ (\sum_{x}{a_{x,y,z}}) * \log{\sum_{x}{k*a_{x,y,z}}} \}}} 
\end{gather*}

\begin{gather*}
\sum_{x}{\sum_{y}{\{ k*(\sum_{z}{a_{x,y,z}}) + (\sum_{z}{a_{x,y,z}}) * \log{\sum_{z}{a_{x,y,z}}} \}}} \leq \\
\leq \sum_{z}{\sum_{y}{\{ k*(\sum_{x}{a_{x,y,z}}) + (\sum_{x}{a_{x,y,z}}) * \log{\sum_{x}{a_{x,y,z}}} \}}} 
\end{gather*}

The first term in inner sum dissappear:
\begin{gather*}
\sum_{x}{\sum_{y}{\{ (\sum_{z}{a_{x,y,z}}) * \log{\sum_{z}{a_{x,y,z}}} \}}} \leq \\
\leq \sum_{z}{\sum_{y}{\{ (\sum_{x}{a_{x,y,z}}) * \log{\sum_{x}{a_{x,y,z}}} \}}} 
\end{gather*}

\begin{gather*}
\sum_{x}{\sum_{y}{\{ (\sum_{z}{a_{x,y,z}}) * \log{\sum_{z}{a_{x,y,z}}} \}}} \leq \\
\leq \sum_{z}{\sum_{y}{\{ (\sum_{z}{a_{x,y,z}}) * \log{\sum_{x}{a_{x,y,z}}} \}}} 
\end{gather*}

We use Golden lemma here for property that (for iteration over $x$) each function is permutation so each value will occur once and
(for iteration over $z$) we will have multiple non-zero elements.

We can observe that Golden Lemma says that euality holds if respective values are equal. In that case it can be implied by having symmetrical matrix
(special cases are discussed below. Such case can correspond to $|S_n|=|L|$ with $X \perp F$)


\subsection{Examples}

We observe that for $X=F$ we alsways have equality in all of the cases.

Now, if we assume $X \perp F$ and $|S_n|=|L|$ then there are $(n-1)!$ permutations for each pair that sends $x$ into $y$:
\begin{equation}
\prob{Y=y}[X=x]=\prob{F(x)=y}[X=x]=\prob{F(x)=y}=1/n
\end{equation}
That is not dependent on $x$ which shows that $Y \perp X$ and we can also show that $Y \perp F$
In that case we have equality, because the mutual information is zeroed.
The first condition implies that this happens for $n=1$ and $n=2$

In more general case for any $n$ and $F=g(X)$ we have equality, because $Y=g(X)(X)$ and both terms are zeroed.

% We can remove sum over $y$ which means that we will quantify over all $y$ for each term:

% \begin{gather*}
% \sum_{x}{\{ (\sum_{z}{a_{x,z}}) * \log{\sum_{z}{a_{x,z}}} \}} \leq \\
% \leq \sum_{z}{\{ (\sum_{x}{a_{x,z}}) * \log{\sum_{x}{a_{x,z}}} \}}
% \end{gather*}

%% END FAKE
  
% I we use the "triangle" inequality i.e $\pre{C}[B] + \pre{B}[A] \geq \pre{C}[A]$ for:
% \begin{equation}
% \begin{cases}
%     A=F\\
%     B=X\\
%     C=Y
% \end{cases}
% \end{equation}

% We get:
% \begin{gather*}
% \pre{Y}[X] + \pre{X}[F] \geq \pre{Y}[F] \\
% - \pre{Y}[X] - \pre{X}[F] \leq - \pre{Y}[F] \\
% \pre{Y} - \pre{Y}[X] - \pre{X}[F] \leq \pre{Y} - \pre{Y}[F] \\
% \pri{Y;X} - \pre{X}[F] \leq \pri{Y;F}
% \end{gather*}

% Doing the same for:
% \begin{equation}
% \begin{cases}
%     A=X\\
%     B=F\\
%     C=Y
% \end{cases}
% \end{equation}
% We get:
% \begin{gather*}
% \pri{Y;F} - \pre{F}[X] \leq \pri{Y;X}
% \end{gather*}

% So:
% \begin{gather*}
% \pri{Y;F} - \pre{F}[X] \leq \pri{Y;X} \\
% \pri{Y;X} - \pre{X}[F] \leq \pri{Y;F}
% \end{gather*}

\if\dataeq{type}{problems}
  \end{document}  
\fi

\if\dataeq{type}{technicalslides}
  \if\dataeq{hasbibliography}{true}
    \begin{frame}
      \frametitle{References}
      % This prints the bibliography on the slide
      \printbibliography
    \end{frame}
  \fi

  \begin{frame}{}
    \centering \Large
    \emph{Thank you}
  \end{frame}
  
  \end{document}  
\fi


% \begin{frame}
%   \frametitle{References}
%   % This prints the bibliography on the slide
%   \printbibliography
% \end{frame}
