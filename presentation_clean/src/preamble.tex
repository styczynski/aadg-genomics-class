\input{vars.tex}
\newcommand{\nop}[1]{}

\def\gettexliveversion#1(#2 #3 #4#5#6#7#8)#9\relax{#4#5#6#7}
\edef\texliveversion{\expandafter\gettexliveversion\pdftexbanner\relax}

\def\bthandled{0}
\def\btunhandled{0}

\if\dataeq{type}{problems}
    \documentclass{article}
    \def\bthandled{1}
    \usepackage{authblk}
    \usepackage[a4paper, total={6in, 8in}]{geometry}
    \author[1]{\data{author}}
    \affil[1]{\data{institute}}
    \date{\data{created}\hspace*{0.5cm}(updated: \data{date})}
    \title{\data{title}}

    \def\buildpdftitle{\data{title}}
    \def\buildpdfsubject{Simple short report for classes}
    \def\buildpdfauthor{\data{author}}
    \def\buildpdfkeywords{}
    \def\buildpdfcreator{\data{author} - \data{institute}}
    \def\buildpdfproducer{TeXLive=\texliveversion; CMake=\data{cmakeversion}; Template=\data{type}(article)}

    \usepackage{lipsum}
\usepackage{tikz}
\usepackage{shellesc}
\usepackage{pdftexcmds}
\usepackage{robust-externalize}
\usepackage{relsize}
\usepackage{algorithm}
\PassOptionsToPackage{noend}{algpseudocode}
\usepackage{algpseudocode}
\usepackage{float}
\usepackage{mathtools}
\usepackage{caption}
\usepackage[dvipsnames, x11names, svgnames]{xcolor}
\usepackage{colortbl}
\usepackage{color,soul}
\usepackage{minted}
\usepackage{filecontents}
\usepackage[backend=biber]{biblatex}


\addbibresource{main.bib}

\pgfmathsetseed{\number\pdfrandomseed}
\usetikzlibrary{tikzmark, positioning, decorations.pathreplacing, calc, arrows.meta, shapes.geometric, backgrounds, arrows}

\robExtConfigure{
  compile in parallel with gnu parallel
}

\def\mathdefault#1{#1} % Needed in matplotlib 3.8: https://github.com/matplotlib/matplotlib/issues/27907
\setbeamertemplate{frametitle}[default][center]
% Removes icon in bibliography
\setbeamertemplate{bibliography item}{}

\renewcommand{\algorithmiccomment}[1]{\hfill$\triangleright$\textit{#1}}
\newcommand{\CommentH}[1]{\Comment{\textbf{\textcolor{BlueViolet}{#1}}}}

\tikzset{fontscale/.style = {font=\relsize{#1}}}

\DeclarePairedDelimiter\ceil{\lceil}{\rceil}
\DeclarePairedDelimiter\floor{\lfloor}{\rfloor}

%% FIXES FOR ALGORITHMS

\usepackage{etoolbox}

\newcommand{\algruledefaultfactor}{.75}
\newcommand{\algstrut}[1][\algruledefaultfactor]{\vrule width 0pt
depth .25\baselineskip height #1\baselineskip\relax}
\newcommand*{\algrule}[1][\algorithmicindent]{\hspace*{.5em}\vrule\algstrut
\hspace*{\dimexpr#1-.5em}}

\makeatletter
\newcount\ALG@printindent@tempcnta
\def\ALG@printindent{%
    \ifnum \theALG@nested>0% is there anything to print
    \ifx\ALG@text\ALG@x@notext% is this an end group without any text?
    % do nothing
    \else
    \unskip
    % draw a rule for each indent level
    \ALG@printindent@tempcnta=1
    \loop
    \algrule[\csname ALG@ind@\the\ALG@printindent@tempcnta\endcsname]%
    \advance \ALG@printindent@tempcnta 1
    \ifnum \ALG@printindent@tempcnta<\numexpr\theALG@nested+1\relax% can't do <=, so add one to RHS and use < instead
    \repeat
    \fi
    \fi
}%

\patchcmd{\ALG@doentity}{\noindent\hskip\ALG@tlm}{\ALG@printindent}{}{\errmessage{failed to patch}}

\AtBeginEnvironment{algorithmic}{\lineskip0pt}

\newcommand*\Let[2]{\State #1 $\gets$ #2}
\newcommand*\Stateh{\State \algstrut[1]}


%% END FIXES
%% COLOURS FOR ALGORITHMS


\makeatletter
\newcommand{\algcolor}[2]{%
  \hskip-\ALG@thistlm\colorbox{#1}{\parbox{\dimexpr\linewidth-2\fboxsep}{\hskip\ALG@thistlm\relax #2}}%
}
\newcommand{\algemph}[1]{\algcolor{GreenYellow}{#1}}
\makeatother

% END COLOURS


\newcommand{\randeq}[1]{% 
\pgfmathparse{(int(random(-40, 40))+100)/100 * #1}%
\pgfmathresult%
}%

\tikzset{
    bigbox/.style={draw, rounded corners, minimum width=1.5cm, minimum height=1cm},
    smallbox/.style={draw, rounded corners, minimum width=1.25cm, minimum height=0.75cm},
    tinybox/.style={draw, rounded corners, minimum width=1.25cm, minimum height=0.6cm},
    bigcircle/.style={draw, circle, minimum size=1cm},
    bigellipse/.style={draw, ellipse, minimum width=1.5cm, minimum height=1.25cm},
    place/.style={inner sep=0pt, outer sep=0pt},
    colprimaryl/.style={draw=NavyBlue, fill=LightSkyBlue, text=black},
    colprimary/.style={draw=NavyBlue, fill=NavyBlue, text=white},
    operation/.style={draw=FireBrick, fill=LightSalmon, text=black, rounded corners},
    fork/.style={decorate, decoration={show path construction, lineto code={
        \draw[->](\tikzinputsegmentfirst)-|($(\tikzinputsegmentfirst)!.5!(\tikzinputsegmentlast)$)|-(\tikzinputsegmentlast);}
    }},
    center coordinate/.style={
        execute at end picture={
        \path ([rotate around={180:#1}]perpendicular cs: horizontal line through={#1},
                                    vertical line through={(current bounding box.east)})
                ([rotate around={180:#1}]perpendicular cs: horizontal line through={#1},
                                    vertical line through={(current bounding box.west)});}}
}

\let\oldtikzpicture\tikzpicture
\let\oldendtikzpicture\endtikzpicture

\renewenvironment{tikzpicture}{%
    \ifbuilddraft\comment%
    \else\expandafter\oldtikzpicture%   
    \fi
}{%
    \ifbuilddraft\endcomment%
    \else\oldendtikzpicture%
    \fi
}


    \begin{document}
    \maketitle
\fi

\if\dataeq{type}{technicalslides}
    \documentclass{beamer}
    \def\bthandled{1}
    \author{\data{author}}
    \institute{\data{institute}}
    \date{\data{created}\hspace*{0.5cm}(updated: \data{date})}
    \title{\data{title}}

    \def\buildpdftitle{\data{title}}
    \def\buildpdfsubject{Technical slide deck}
    \def\buildpdfauthor{\data{author}}
    \def\buildpdfkeywords{}
    \def\buildpdfcreator{\data{author} - \data{institute}}
    \def\buildpdfproducer{TeXLive=\texliveversion; CMake=\data{cmakeversion}; Template=\data{type}(beamer)}

    \if\dataeq{hasbibliography}{true}
        \usepackage[backend=biber, style=authortitle]{biblatex}
        \addbibresource{main.bib}
        \AtEveryCitekey{\iffootnote{\scriptsize}{\footnotesize}}
        \setbeamertemplate{navigation symbols}{}
    \fi
    \setbeamertemplate{frametitle}[default][center]

    \usepackage{lipsum}
\usepackage{tikz}
\usepackage{shellesc}
\usepackage{pdftexcmds}
\usepackage{robust-externalize}
\usepackage{relsize}
\usepackage{algorithm}
\PassOptionsToPackage{noend}{algpseudocode}
\usepackage{algpseudocode}
\usepackage{float}
\usepackage{mathtools}
\usepackage{caption}
\usepackage[dvipsnames, x11names, svgnames]{xcolor}
\usepackage{colortbl}
\usepackage{color,soul}
\usepackage{minted}
\usepackage{filecontents}
\usepackage[backend=biber]{biblatex}


\addbibresource{main.bib}

\pgfmathsetseed{\number\pdfrandomseed}
\usetikzlibrary{tikzmark, positioning, decorations.pathreplacing, calc, arrows.meta, shapes.geometric, backgrounds, arrows}

\robExtConfigure{
  compile in parallel with gnu parallel
}

\def\mathdefault#1{#1} % Needed in matplotlib 3.8: https://github.com/matplotlib/matplotlib/issues/27907
\setbeamertemplate{frametitle}[default][center]
% Removes icon in bibliography
\setbeamertemplate{bibliography item}{}

\renewcommand{\algorithmiccomment}[1]{\hfill$\triangleright$\textit{#1}}
\newcommand{\CommentH}[1]{\Comment{\textbf{\textcolor{BlueViolet}{#1}}}}

\tikzset{fontscale/.style = {font=\relsize{#1}}}

\DeclarePairedDelimiter\ceil{\lceil}{\rceil}
\DeclarePairedDelimiter\floor{\lfloor}{\rfloor}

%% FIXES FOR ALGORITHMS

\usepackage{etoolbox}

\newcommand{\algruledefaultfactor}{.75}
\newcommand{\algstrut}[1][\algruledefaultfactor]{\vrule width 0pt
depth .25\baselineskip height #1\baselineskip\relax}
\newcommand*{\algrule}[1][\algorithmicindent]{\hspace*{.5em}\vrule\algstrut
\hspace*{\dimexpr#1-.5em}}

\makeatletter
\newcount\ALG@printindent@tempcnta
\def\ALG@printindent{%
    \ifnum \theALG@nested>0% is there anything to print
    \ifx\ALG@text\ALG@x@notext% is this an end group without any text?
    % do nothing
    \else
    \unskip
    % draw a rule for each indent level
    \ALG@printindent@tempcnta=1
    \loop
    \algrule[\csname ALG@ind@\the\ALG@printindent@tempcnta\endcsname]%
    \advance \ALG@printindent@tempcnta 1
    \ifnum \ALG@printindent@tempcnta<\numexpr\theALG@nested+1\relax% can't do <=, so add one to RHS and use < instead
    \repeat
    \fi
    \fi
}%

\patchcmd{\ALG@doentity}{\noindent\hskip\ALG@tlm}{\ALG@printindent}{}{\errmessage{failed to patch}}

\AtBeginEnvironment{algorithmic}{\lineskip0pt}

\newcommand*\Let[2]{\State #1 $\gets$ #2}
\newcommand*\Stateh{\State \algstrut[1]}


%% END FIXES
%% COLOURS FOR ALGORITHMS


\makeatletter
\newcommand{\algcolor}[2]{%
  \hskip-\ALG@thistlm\colorbox{#1}{\parbox{\dimexpr\linewidth-2\fboxsep}{\hskip\ALG@thistlm\relax #2}}%
}
\newcommand{\algemph}[1]{\algcolor{GreenYellow}{#1}}
\makeatother

% END COLOURS


\newcommand{\randeq}[1]{% 
\pgfmathparse{(int(random(-40, 40))+100)/100 * #1}%
\pgfmathresult%
}%

\tikzset{
    bigbox/.style={draw, rounded corners, minimum width=1.5cm, minimum height=1cm},
    smallbox/.style={draw, rounded corners, minimum width=1.25cm, minimum height=0.75cm},
    tinybox/.style={draw, rounded corners, minimum width=1.25cm, minimum height=0.6cm},
    bigcircle/.style={draw, circle, minimum size=1cm},
    bigellipse/.style={draw, ellipse, minimum width=1.5cm, minimum height=1.25cm},
    place/.style={inner sep=0pt, outer sep=0pt},
    colprimaryl/.style={draw=NavyBlue, fill=LightSkyBlue, text=black},
    colprimary/.style={draw=NavyBlue, fill=NavyBlue, text=white},
    operation/.style={draw=FireBrick, fill=LightSalmon, text=black, rounded corners},
    fork/.style={decorate, decoration={show path construction, lineto code={
        \draw[->](\tikzinputsegmentfirst)-|($(\tikzinputsegmentfirst)!.5!(\tikzinputsegmentlast)$)|-(\tikzinputsegmentlast);}
    }},
    center coordinate/.style={
        execute at end picture={
        \path ([rotate around={180:#1}]perpendicular cs: horizontal line through={#1},
                                    vertical line through={(current bounding box.east)})
                ([rotate around={180:#1}]perpendicular cs: horizontal line through={#1},
                                    vertical line through={(current bounding box.west)});}}
}

\let\oldtikzpicture\tikzpicture
\let\oldendtikzpicture\endtikzpicture

\renewenvironment{tikzpicture}{%
    \ifbuilddraft\comment%
    \else\expandafter\oldtikzpicture%   
    \fi
}{%
    \ifbuilddraft\endcomment%
    \else\oldendtikzpicture%
    \fi
}


    \begin{document}
    \frame{\titlepage}
\fi


\ifx\bthandled\btunhandled
    % Throw error normally
    \PackageError{preamble}{Unrecognized type of document: '\data{type}'}{}
    % change interaction mode (if not yet in batchmode)
    \batchmode
    % ask for terminal input (which would be stored in the macro `\foo`)
    \read-1 to \foo
\fi
