\documentclass{article}

\input{vars.tex}
\usepackage{lipsum}
\usepackage{tikz}
\usepackage{shellesc}
\usepackage{pdftexcmds}
\usepackage{robust-externalize}
\usepackage{relsize}
\usepackage{algorithm}
\PassOptionsToPackage{noend}{algpseudocode}
\usepackage{algpseudocode}
\usepackage{float}
\usepackage{mathtools}
\usepackage{caption}
\usepackage[dvipsnames, x11names, svgnames]{xcolor}
\usepackage{colortbl}
\usepackage{color,soul}
\usepackage{minted}
\usepackage{filecontents}
\usepackage[backend=biber]{biblatex}


\addbibresource{main.bib}

\pgfmathsetseed{\number\pdfrandomseed}
\usetikzlibrary{tikzmark, positioning, decorations.pathreplacing, calc, arrows.meta, shapes.geometric, backgrounds, arrows}

\robExtConfigure{
  compile in parallel with gnu parallel
}

\def\mathdefault#1{#1} % Needed in matplotlib 3.8: https://github.com/matplotlib/matplotlib/issues/27907
\setbeamertemplate{frametitle}[default][center]
% Removes icon in bibliography
\setbeamertemplate{bibliography item}{}

\renewcommand{\algorithmiccomment}[1]{\hfill$\triangleright$\textit{#1}}
\newcommand{\CommentH}[1]{\Comment{\textbf{\textcolor{BlueViolet}{#1}}}}

\tikzset{fontscale/.style = {font=\relsize{#1}}}

\DeclarePairedDelimiter\ceil{\lceil}{\rceil}
\DeclarePairedDelimiter\floor{\lfloor}{\rfloor}

%% FIXES FOR ALGORITHMS

\usepackage{etoolbox}

\newcommand{\algruledefaultfactor}{.75}
\newcommand{\algstrut}[1][\algruledefaultfactor]{\vrule width 0pt
depth .25\baselineskip height #1\baselineskip\relax}
\newcommand*{\algrule}[1][\algorithmicindent]{\hspace*{.5em}\vrule\algstrut
\hspace*{\dimexpr#1-.5em}}

\makeatletter
\newcount\ALG@printindent@tempcnta
\def\ALG@printindent{%
    \ifnum \theALG@nested>0% is there anything to print
    \ifx\ALG@text\ALG@x@notext% is this an end group without any text?
    % do nothing
    \else
    \unskip
    % draw a rule for each indent level
    \ALG@printindent@tempcnta=1
    \loop
    \algrule[\csname ALG@ind@\the\ALG@printindent@tempcnta\endcsname]%
    \advance \ALG@printindent@tempcnta 1
    \ifnum \ALG@printindent@tempcnta<\numexpr\theALG@nested+1\relax% can't do <=, so add one to RHS and use < instead
    \repeat
    \fi
    \fi
}%

\patchcmd{\ALG@doentity}{\noindent\hskip\ALG@tlm}{\ALG@printindent}{}{\errmessage{failed to patch}}

\AtBeginEnvironment{algorithmic}{\lineskip0pt}

\newcommand*\Let[2]{\State #1 $\gets$ #2}
\newcommand*\Stateh{\State \algstrut[1]}


%% END FIXES
%% COLOURS FOR ALGORITHMS


\makeatletter
\newcommand{\algcolor}[2]{%
  \hskip-\ALG@thistlm\colorbox{#1}{\parbox{\dimexpr\linewidth-2\fboxsep}{\hskip\ALG@thistlm\relax #2}}%
}
\newcommand{\algemph}[1]{\algcolor{GreenYellow}{#1}}
\makeatother

% END COLOURS


\newcommand{\randeq}[1]{% 
\pgfmathparse{(int(random(-40, 40))+100)/100 * #1}%
\pgfmathresult%
}%

\tikzset{
    bigbox/.style={draw, rounded corners, minimum width=1.5cm, minimum height=1cm},
    smallbox/.style={draw, rounded corners, minimum width=1.25cm, minimum height=0.75cm},
    tinybox/.style={draw, rounded corners, minimum width=1.25cm, minimum height=0.6cm},
    bigcircle/.style={draw, circle, minimum size=1cm},
    bigellipse/.style={draw, ellipse, minimum width=1.5cm, minimum height=1.25cm},
    place/.style={inner sep=0pt, outer sep=0pt},
    colprimaryl/.style={draw=NavyBlue, fill=LightSkyBlue, text=black},
    colprimary/.style={draw=NavyBlue, fill=NavyBlue, text=white},
    operation/.style={draw=FireBrick, fill=LightSalmon, text=black, rounded corners},
    fork/.style={decorate, decoration={show path construction, lineto code={
        \draw[->](\tikzinputsegmentfirst)-|($(\tikzinputsegmentfirst)!.5!(\tikzinputsegmentlast)$)|-(\tikzinputsegmentlast);}
    }},
    center coordinate/.style={
        execute at end picture={
        \path ([rotate around={180:#1}]perpendicular cs: horizontal line through={#1},
                                    vertical line through={(current bounding box.east)})
                ([rotate around={180:#1}]perpendicular cs: horizontal line through={#1},
                                    vertical line through={(current bounding box.west)});}}
}

\let\oldtikzpicture\tikzpicture
\let\oldendtikzpicture\endtikzpicture

\renewenvironment{tikzpicture}{%
    \ifbuilddraft\comment%
    \else\expandafter\oldtikzpicture%   
    \fi
}{%
    \ifbuilddraft\endcomment%
    \else\oldendtikzpicture%
    \fi
}



\usepackage{blindtext}
\usepackage[a4paper, total={6in, 8in}]{geometry}

\makeatletter
\newcommand{\defeq}{\vcentcolon=}
\newcommand{\@giventhatstar}[2]{\left(#1\;\middle|\;#2\right)}
\newcommand{\@giventhatnostar}[3][]{#1(#2\;#1|\;#3#1)}
\newcommand{\giventhat}{\@ifstar\@giventhatstar\@giventhatnostar}
\newcommand\problike[2]{\def\probname{#1}\def\probp1{#2}\problikeaux}
\newcommand\problikeaux[1][]{%
  \ifx\relax#1\relax%
    \probname(\probp1)
  \else%
    \probname\giventhat{\probp1}{#1}
  \fi
}
\newcommand\prob{\problike{Pr}}
\newcommand\pre{\problike{H}}
\newcommand\pri{\problike{I}}
\newcommand\prt[2]{\problike{P}{#1 \rightarrow #2}}
\newcommand\prtf[3]{\problike{P_{#1}}{#2 \rightarrow #3}}
\DeclareRobustCommand\iff{\;\Longleftrightarrow\;}
\makeatother

\newcommand\bigforall{%
  \mathop{\lower0.75ex\hbox{\ensuremath{%
    \mathlarger{\mathlarger{\mathlarger{\mathlarger{\forall}}}}}}}%
  \limits}

\newcommand\bigexists{%
  \mathop{\lower0.75ex\hbox{\ensuremath{%
    \mathlarger{\mathlarger{\mathlarger{\mathlarger{\exists}}}}}}}%
  \limits}

\title{Teoria informacji kolokwium 4 grudnia 2024}

\begin{document}

\maketitle

\section{Problem 2}

\subsection{Task 1}

We observe that for binary input and perfect channel combined in series with Z-channel we have:
\begin{equation}
\begin{cases}
    \prtf{A}{\alpha}{\alpha} = 1 \\
    \prtf{A}{\beta}{\beta} = 1 \\
    \prtf{B}{\alpha}{\alpha} = 1-p \\
    \prtf{B}{\alpha}{\beta} = p \\
    \prtf{B}{\beta}{\beta} = 1
\end{cases}
\end{equation}

$\prob{B}[A,C]$ won't be well defined for a channel matrix, because:
\begin{equation}
\begin{cases}
\prob{B=\alpha}[A=\alpha, C=\beta] = 1 \\
\prob{B=\alpha}[A=\alpha, C=\alpha] = 1
\end{cases}
\end{equation}

Hence the requirement that there are no zero elements in matrices of both channels will come into play.
We consider the probability to be well defined if:

\begin{enumerate}
\item $\bigexists_{f \perp P(A)}{\bigforall_{\substack{a \in A \\ c \in C}}{ \prob{B=b}[A=a, C=c] = f(...) }}$
\item $\bigforall_{\substack{a \in A \\ c \in C}}{ 0 < \prob{B=b}[A=a, C=c] < 1 }$
\item $\bigforall_{\substack{a \in A \\ c \in C}}{ \sum_{b \in B}{\prob{B=b}[A=a, C=c]} = 1 }$
\end{enumerate}

Now let's start by expanding the probability using Bayes' theorem:
\begin{equation}
\prob{b}[a,c] = \frac{ \prob{a,c}[b]*\prob{b} }{ \prob{a,c} }
\end{equation}

For numerator we use the fact that $\prob{x,y}[z] = \prob{x}[y,z]*\prob{y}[z]$ and for enumerator we use $\prob{x,y}=\prob{x}[y]*\prob{y}$:
\begin{equation}
\prob{b}[a,c] = \frac{ \prob{a}[b,c]*\prob{b}*\prob{c}[b] }{ \prob{a}[c] * \prob{c} }
\end{equation}

Now we use conditional independence contraint so $\prob{a}[b,c] = \prob{a}[b]$:
\begin{equation}
\prob{b}[a,c] = \frac{ \prob{a}[b]*\prob{b}*\prob{c}[b] }{ \prob{a}[c] * \prob{c} }
\end{equation}

We use the fact that $\prob{x}[y] = \prob{y}[x] * \frac{\prob{x}}{\prob{y}}$:
\begin{equation}
\prob{b}[a,c] = \frac{ \prob{b}[a]*\prob{a}*\prob{b}*\prob{c}[b] }{ \prob{a}[c] * \prob{c} * \prob{b} }
\end{equation}

After reorganising terms:
\begin{equation}
\prob{b}[a,c] = (\prob{b}[a] * \prob{c}[b]) * \frac{ \prob{a} }{ \prob{a}[c] } * \frac{1}{\prob{c}}
\end{equation}

Now let's observe that $\frac{\prob{x}}{\prob{x}[y]} = \frac{\prob{y}}{\prob{y}[x]}$:
\begin{equation}
\prob{b}[a,c] = (\prob{b}[a] * \prob{c}[b]) * \frac{ \prob{c} }{ \prob{c}[a] } * \frac{1}{\prob{c}}
\end{equation}

This can be written as:
\begin{equation}
\prob{b}[a,c] = \prtf{\Gamma_1}{a}{b} * \prtf{\Gamma_2}{b}{c} * \frac{ 1 }{ \prob{c}[a] }
\end{equation}

We observe that $\prob{c}[a]$ in fact does not depend on the distribution of $A$, which proves first point.
In all our transformations we inmplicitly use the facts that $\prob{a}, \prob{b}, \prob{c} \neq 0$
Second point is trivial to show, because $0 < \prtf{\Gamma_1}{a}{b}, \prtf{\Gamma_2}{b}{c}, \prob{c}[a] < 1$
Third point is direct consequence of conditional independence.

\subsection{Task 2}

Let's assume we take channel $\Gamma_E$ and squeeze the symbol space using mapping $s_1 \in {(A \times C)}^{A} \defeq s_1(a,c) = a$

In such scenario:
\begin{gather*}
\prtf{\Gamma_E}{s_1(a,c)}{b} = \sum_{c}{ \prtf{\Gamma_E}{a,c}{b} } \\
\sum_{c}{ \prtf{\Gamma_1}{a}{b} * \prtf{\Gamma_2}{b}{c} * \frac{ 1 }{ \prob{c}[a] } } \\
\prtf{\Gamma_F}{a}{b} = \prtf{\Gamma_1}{a}{b}  * \sum_{c}{ \frac{ \prtf{\Gamma_2}{b}{c} }{ \prob{c}[a] } } 
%(\Gamma_E)_{s_1(a,b), c} = \frac{ 1 }{ \prob{c}[a] } * (\Gamma_1 \times \Gamma_2)_{a,c}
\end{gather*}

We know that $\sum_{c}{\prob{c}[a]} = 1$

If we compare $\prtf{\Gamma_F}{a}{b}$ and $\prtf{\Gamma_1}{a}{b}$ then in $I(A,B)$ we will notice that in worst case we can select such a coefficients that $\Gamma_F$ is as bad as using $\Gamma_1$
The original matrix $\Gamma_F$ has more rows than $\Gamma_E$ (such that rows of $\Gamma_F$ are linear combinations of $\Gamma_E$) meaning that the capacity of original $Gamma_E$ cannot be lower.

Hance we prove that capacity of $\Gamma_E$ is no lower than $\Gamma_F$

\subsection{Task 3}

My best guess is that we use some prime factors to make $\Gamma_E$ have all rows independent, then that will work. However I was unable to showcase some concrete example.

\end{document}
